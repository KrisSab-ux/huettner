\tcbset{
  userstory/.style={
    colback=blue!5,
    colframe=blue!75!black,
    boxrule=0.8mm,
    arc=3mm,
    fonttitle=\bfseries,
    coltitle=black,
    title={#1},
    left=3mm,
    right=3mm,
    top=2mm,
    bottom=2mm,
  }
}

%------------------------------------------------------------------------------------------------------------------------------------------------------------
\section{User Story Cards und Epics}
\begin{tcolorbox}[userstory={Key/Value Dictionary erstellen}]
\begin{tabular}{|m{2cm}|m{6cm}|m{2cm}|}
\hline
\textbf{Prio} & \textbf{Name} & \textbf{Points}\\
\hline
1 & Key/Value Dictionary erstellen & 30h \\
\hline
\end{tabular}

\vspace{2mm}

\textbf{Als} EntwicklerIn \\
\textbf{möchte ich} aus einem eigens zu diesem Zweck generierten Datensatz die wichtigsten und gängigsten Key-Value Paare für Start der Software extrahieren \\
\textbf{um} um einen repräsentativen Start und Default für die Software bereitszustellen \\

\vspace{2mm}

\begin{tabular}{|m{3cm}|m{7cm}|}
\hline
\textbf{Risiko} & [Beschreibung] \\
\hline
\textbf{Points (Post)} &  \\
\hline
\end{tabular}

\vspace{2mm}

\textbf{Vorausgesetzt} ein solcher Datensatz konnte generiert oder beschaffen werden \\
\textbf{wenn} die Software Dokumente per OCR ausliest, \textbf{dann} wird ein festgeleger Prozentsatz der Dokumente anhand bestehender Beispiele korrekt zugeordnet
\end{tcolorbox}
%------------------------------------------------------------------------------------------------------------------------------------------------------------

\vspace{1cm}

%------------------------------------------------------------------------------------------------------------------------------------------------------------
\begin{tcolorbox}[userstory={Training des Modells}]
\begin{tabular}{|m{2cm}|m{6cm}|m{2cm}|}
\hline
\textbf{Prio} & \textbf{Name} & \textbf{Points} \\
\hline
1 & Trainng des Modells & /h \\
\hline
\end{tabular}

\vspace{2mm}

\textbf{Als} EntwicklerIn \\
\textbf{möchte ich} aus den festgelegten Trainingsdaten ein Modell trainieren, welches Rechnungen inhaltlich den korrekten Buchungskonten zuweisen kann\\
\textbf{um} das Bearbeiten von Rechnungen zu beschleunigen und Fehler durch weniger anfallende manuelle und monotone Arbeit zu reduzieren \\

\vspace{2mm}

\begin{tabular}{|m{3cm}|m{7cm}|}
\hline
\textbf{Risiko} & [Beschreibung] \\
\hline
\textbf{Points (Post)} & 3 \\
\hline
\end{tabular}

\vspace{2mm}

\textbf{Vorausgesetzt} das Modell erhält repräsentative Daten, \\
\textbf{wenn} die Daten korrekt geesplittet wurden, \textbf{dann} können erste korrekte Klassifizierungen neuer Rechnungen durch das Training vorgenommen werden.
\end{tcolorbox}
%------------------------------------------------------------------------------------------------------------------------------------------------------------

\vspace{1cm}

%------------------------------------------------------------------------------------------------------------------------------------------------------------
\begin{tcolorbox}[userstory={Modell evaluieren}]
\begin{tabular}{|m{2cm}|m{6cm}|m{2cm}|}
\hline
\textbf{Prio} & \textbf{Name} & \textbf{Points} \\
\hline
2 & Modell evaluieren & 32h \\
\hline
\end{tabular}

\vspace{2mm}

\textbf{Als} EntwicklerIn \\
\textbf{möchte ich} das trainierte Modell mit Testdaten evaluieren \\
\textbf{um} festzustellen, wie genau es neuen Rechnungen die korrekte Hauptkonten zuordnen kann \\

\vspace{2mm}

\begin{tabular}{|m{3cm}|m{7cm}|}
\hline
\textbf{Risiko} & [Beschreibung] \\
\hline
\textbf{Points (Post)} & 3 \\
\hline
\end{tabular}

\vspace{2mm}

\textbf{Vorausgesetzt} das Modell ist trainiert und einsatzfähig, \\
\textbf{wenn} eine Menge von mindestens 100 Testrechnungen verarbeitet wird, \textbf{dann}, werden erste korrekte Rechnungsklassifikationen erstellet und ausgegeben.
\end{tcolorbox}
%------------------------------------------------------------------------------------------------------------------------------------------------------------

\vspace{1cm}

%------------------------------------------------------------------------------------------------------------------------------------------------------------
\begin{tcolorbox}[userstory={Modell integrieren}]
\begin{tabular}{|m{2cm}|m{6cm}|m{2cm}|}
\hline
\textbf{Prio} & \textbf{Name} & \textbf{Points} \\
\hline
3 & Modell integrieren & 48h \\
\hline
\end{tabular}

\vspace{2mm}

\textbf{Als} EntwicklerIn \\
\textbf{möchte ich} das trainierte Modell in die Document Analyzer Pipeline integrieren \\
\textbf{um} es Benutzern beim Uplaod automatisch die exxtrahierten Daten zeigen lassen zu können \\

\vspace{2mm}

\begin{tabular}{|m{3cm}|m{7cm}|}
\hline
\textbf{Risiko} & [Beschreibung] \\
\hline
\textbf{Points (Post)} & 3 \\
\hline
\end{tabular}

\vspace{2mm}

\textbf{Vorausgesetzt} das Modell ist erfolgreich trainiert und getestet, \\
\textbf{wenn} der Benutzer Rechnungen hochlädt, \textbf{dann},  wird das Modell automatisch aufgerufen und liefert JSON-Ausgaben mit Schlüssel/Wert-Paaren welcher als Outputin im UI angezeogt wird. Cache wird nach Bestätigung DSGVO-konform gelöscht.
\end{tcolorbox}
%------------------------------------------------------------------------------------------------------------------------------------------------------------

\vspace{1cm}

%------------------------------------------------------------------------------------------------------------------------------------------------------------
\begin{tcolorbox}[userstory={Modell überwachen und verbessern}]
\begin{tabular}{|m{2cm}|m{6cm}|m{2cm}|}
\hline
\textbf{Prio} & \textbf{Name} & \textbf{Points} \\
\hline
4 & Modell überwachen und verbessern & 32h \\
\hline
\end{tabular}

\vspace{2mm}

\textbf{Als} EntwicklerIn \\
\textbf{möchte ich} die Integrität meines Modells auf die bestmögliche Performance verbessern\\
\textbf{um} die Fehlerrate zu minimieren und User zuverlässige Ergebnisse zu präsentieren\\

\vspace{2mm}

\begin{tabular}{|m{3cm}|m{7cm}|}
\hline
\textbf{Risiko} & [Beschreibung] \\
\hline
\textbf{Points (Post)} & 3 \\
\hline
\end{tabular}

\vspace{2mm}

\textbf{Vorausgesetzt} das Modell läuft und ist stabil, \\
\textbf{wenn} permanent Rechnungen zur Klassifikation hochgeladen werden, \textbf{dann}, werden Hyperparameter so verändert, dass F1-Score und Accuracy bei mindestens 98 Prozent liegen.
\end{tcolorbox}
%------------------------------------------------------------------------------------------------------------------------------------------------------------
