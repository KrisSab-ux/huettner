\section{Projektbeschreibung}


Im Projekt Document Analyzer sollen Rechnungen verschiedener, on digital oder gescannt, maschinell ausgelesen und einem der Hauptkonten des Buchungswesen zugeordnet werden können. 
Semi-überwachte maschinelle Lernparadigma (Semi-Supervised Learning) sowie moderne Techniken aus dem Bereich Computer Vision und weiteren modernen Verfahren werden genutzt um ein Modell zu kreieren, welches
Schlüsselinformationen wie Rechnungsnummer, Betreff und verwendungszweck, Datum, Betrag und Absender aus Rechnungen extrahieren kann um dieses Ziel zu erreichen.
Dafür werden Trainingsdaten benötigt, bei denen diese Informationen bereits korrekt annotiert sind. 
Das Modell lernt also aus Beispielen, welche Textstellen zu welchen Informationskategorien gehören. Dieses Vorgehen ermöglicht eine gezielte Leistungsbewertung anhand bekannter Zielwerte (Labels) und unterstützt eine iterative Verbesserung der Extraktionsgenauigkeit. 
Das semi-überwachte Lernen ist daher ideal geeignet, um eine hohe Präzision und Nachvollziehbarkeit in der Dokumentenanalyse sicherzustellen. 